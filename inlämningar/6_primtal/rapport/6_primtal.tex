\documentclass[a4paper,11pt]{article}

\usepackage[utf8]{inputenc}

\usepackage{graphicx}

\usepackage{minted}

\usepackage{semantic}

\begin{document}

\title{
  \textbf{Primtal\\
  \small Programmering II}
}
\author{Axel Karlsson}
\date{VT22 \today}

\maketitle

\section*{Inledning}
I denna uppgift har jag skapat tre funktioner som på olika sätt generar en lista av alla primtal från 2 till ett givet tal. Jag har sedan jämfört körtiden av funktionerna.

\section*{Generera primtal}
I alla tre versioner använde jag även en eller flera hjälpfunktioner.
\subsection*{Version 1}
För att implementera denna funktion använde jag en hjälpfunktion {\tt rem\_el()}. Den tar en lista och ett element som argument, och sedan tar den bort alla tal som är delbarar med elementet. Denna funktion anropas på alla tal i listan med tal från 2 till {\tt n}.

\begin{minted}{elixir}
  def prim1([]) do [] end
  # Listan innehåller alla tal [2, n]
  def prim1([h | t]) do [h | prim1(rem_el(h, t))] end # Gå genom alla element

  rem_el(_, []) do [] end
  rem_el(e, [h | t]) do
  case rem(h, e) do
    0 -> rem_el(e, t) # Ta bort elementet ur listan
    _ -> [h | rem(e, t)] # Spara elementet i listan
  end
\end{minted}

\subsection*{Version 2}
Denna version använder en ackumulatorlista av primtal och två hjälpfunktioner. {\tt check\_lst()} kollar om ett element är med i en lista. {\tt add\_lst()} lägger till ett element sist i en lista.
\begin{minted}{elixir}
  # Börja med 2 i ackumulatorlistan
  def prim2(n) when is_integer(n) do prim2(Enum..., [2])
  
  def prim2([], prims) do prims end
  def prim2([h  t], prims) do
    case check:lst(h, prims) ->
      :succ -> prim2(t, add_lst(h, prims)) # Spara element
      :fail -> prim2(t, prims) # Släng element
    end
  end
  
  def add_last(e, [h]) do [h | [e]] end
  def add_last(e, [h | t]) do [h | add_last(e, t)] end
\end{minted}

\subsection*{Version 3}
Denna version fungerar på samma sätt som version 2, men istället för att lägga till varje primtal sist i ackumulatorlistan vänder vi på den när vi är klara.
Detta görs med funktionen {\tt rev\_lst()}
\begin{minted}{elixir}
  # Funktionen för att vända p listan
  def lst_rev(lst) do lst_rev(lst, []) # Använder en ackumulatorlista
  def lst_rev([], rev) do rev end # Vi ha gått igenom alla element
  def lst_rev([h | t], rev) do lst_rev(t, [h | rev]) end
\end{minted}

\section*{Benchmark}
Jag skapade en enkel benchmarkfunktion för att benchmarka mina funktioner.

\begin{minted}{elixir}
  def measure(function) do
    function
    |> :timer.tc
    |> elem(0)
    end
  \end{minted}
  
\begin{figure}
  \includegraphics[width=\linewidth]{./bench/bench.png}
  \caption{Resultaten från {\tt measure()}}
  \label{fig:bench_all}
\end{figure}

  I \ref{fig:bench_all} ser vi resultatet från alla tre versioner av {\tt prim()} då de kördes i {\tt measure()}.

  \pagebreak
\section*{Slutsatser}
Som vi ser i \ref{fig:bench_all} var version 1 \& 2 av {\tt prim()} ungefär lika snabba, och version 3 var långsammare. Jag blev överraskad av detta resultat, innan jag testade dem trodde jag att version 1 skulle vara snabbast och att version 2 \& 3 skulle vara ungefär lika snabba. \\
Efter att ha reflekterat mer över dessa resultat kan jag förstå varför version 1 \& 2 är ungefär lika snabba, då de måste gå samma antal ``steg'' i listor men gör det på olika sätt. Jag är inte säker på varför version 3 är så pass mycket långsammare, om jag behövde ge en anledning skulle jag säga att det beror på att det är långsamt att vända en lista, men jag är inte säker på varför det är så mycket långsammare än tillvägagångssättet version 2 använder.

\end{document}
