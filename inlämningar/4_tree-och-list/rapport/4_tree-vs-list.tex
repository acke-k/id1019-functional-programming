\documentclass[a4paper,11pt]{article}

\usepackage[utf8]{inputenc}

\usepackage{graphicx}

\usepackage{minted}

\usepackage{mathtools}

\begin{document}

\title{
  \textbf{List vs. tree\\
  \small Programmering II}
}
\author{Axel Karlsson}
\date{\today}

\maketitle

\section*{Inledning}
I denna rapport ska jag redovisa mitt arbete för uppgiften ``tree vs list'' i kursen programmering II. Jag har implementerat ett träd och en lista, och sedan jämfört deras prestanda i den givna funktionen {\tt bench()}.

\section*{Listan och trädet}
Jag implementerade listan som i koden nedanför.
\begin{minted}{elixir}
  def list_new() do [] end
  
  def list_insert(e, []) do [e] end
  def list_insert(e, [h|t]) do
    case e < h do
      true -> [e | [h | t]]
      false -> [h | list_insert(e, t)]
    end
  end
\end{minted}
Funktionen {\tt list\_add()} springer genom listan tills den hittar ett element som är större än det som ska stoppas in, och länkar in det på den platsen.\\
För trädet tog jag inspiration från förra veckans föreläsning och skrev koden nedan.
\begin{minted}{elixir}
  def tree_new() do :nil end

  def tree_insert(e, :nil) do {e, :nil, :nil} end
  def tree_insert(e, {a, l ,r}) do
    case e < a do
      true -> {a, tree_insert(e, l), r}
      false -> {a, l, tree_insert(e, r)}
    end
  end
\end{minted}
Jag använde även en dummyfunktion som baseline. Den implementerades med koden nedan.
\begin{minted}{elixir}
  def dummy_new() do end
  def dummy_insert(_, _) do end
\end{minted}

\section*{Benchmark}
Benchmarket gjordes med {\tt bench()} funktionen som var given i uppgiften. Jag använde den uppdaterade versionen som körde {\tt bench()} hundra gånger, och sedan lade jag till kod som dividerade resultatet med hundra.
\begin{figure}[H]
  \center
  \includegraphics[]{bench/bench_log.pdf}
  \caption{Resultaten från bench med tree, list och dummy. Logaritmisk y-axel.}
  \label{bench_log}
\end{figure}

\noindent I \ref{bench_log} kan vi se att trädet är mycket snabbare än listan. Observera att y-axeln är logaritmisk. I \ref{bench_linj} plottar vi samma data mot en linjär y-axel och då kan vi knappt se tree-och dummyfunktionen.

\begin{figure}[H]
  \center
  \includegraphics[]{bench/bench_linj.pdf}
  \caption{Resultaten ur bench med tree, list och dummy. Linjär y-axel.}
  \label{bench_linj}
\end{figure}
\section*{Analys}
%Vilka slutsatser kan man dra?
Som vi kan se i figurerna ovan presterar trädet mycket bättre än listan i {\tt bench()}.\\
Funktionen {\tt bench()} fungerar så att {\tt insert()} funktionen för varje typ (list, tree och dummy) anropas så många gånger som y-axeln visar, med ett slumpmässigt nummer mellan noll och hundratusen som argument. Alltså är det {\tt tree\_insert()} som är snabbare än {\tt list\_insert()}. Vi kan förstår varför vi får detta resultat genom att gå igenom vad som händer när vi anroper dem.\\
Om vi kollar på koden för {\tt list\_insert()} ser vi att den först kommer springa igenom listan tills villkoret \(e < h\) uppfylls, vilket i genomsnitt blir \(N/2\) ``steg'' i listan. Elementet {\tt e} kommer då länkas in. När vi sedan går tillbaka genom rekursionen kommer vi att bygga listan i varje anrop av {\tt list\_insert()}, vilket innebär att vi i genomsnitt tar ytterligare \(N/2\) steg per rekursivt anrop. Detta kulminerar i att komplexiteten blir kvadratisk.

\begin{figure}[H]
  \center
  \includegraphics[]{bench/bench_list_fun.pdf}
  \caption{Resultaten ur bench med list samt funktionen \(x^2\).}
  \label{bench_list_fun}
\end{figure}

\noindent I \ref{bench_list_fun} kan vi se att exekveringstiden av {\tt bench()} som en funktion av antalet element \(N\) växer i ungefär samma takt som \(N^2\).\\

\noindent Vi betraktar {\tt tree\_insert()}. Eftersom de element vi lägger till är slumpmässiga kan vi anta att trädet är någorlunda balanserat. Detta innebär att trädet blir \(log_2(N)\) djupt och att vi i genomsnitt måste gå \(log_2(N)/2\) många steg genom trädet för att lägga till ett element. I varje rekursion måste vi sedan bygga trädet igen, vilket resulterar i \(N*log(N)\) komplexitet.
\begin{figure}[H]
  \center
  \includegraphics[]{bench/bench_tree_fun.pdf}
  \caption{Resultaten ur bench med tree samt funktionen \(x * log(x)\).}
  \label{bench_tree_fun}
\end{figure}
\noindent I \ref{bench_tree_fun} kan vi se att tiden från {\tt bench()} växer i ungeför samma takt som \(N*log(N)\).

% Kanske ha benchmark som undersöker hur djupt trädet är.
\section*{Slutsatser}
När jag såg hur stor skillnaden var mellan listan och trädet var det första jag tänkte att jag borde ha använt ett träd istället för en lista i den förra uppgiften ``mips emulator'' när jag implementerade minnet. Denna uppgift var en viktig påminnnelse på det jag lärde mig i kursen data \& algoritmer, och det är roligt att se denna kunskap komma till användning i en annan kurs.

\end{document}
